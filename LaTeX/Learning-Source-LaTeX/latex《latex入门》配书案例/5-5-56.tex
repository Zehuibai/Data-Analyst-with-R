\begin{figure}
\centering
% \usepackage{siunitx}
\begin{tikzpicture}[scale=1.5] % 整体放大坐标,但不影响字号
  \newcommand\iangle{120}
  % 左边的单位圆
  \begin{scope}[xshift=-2cm]
    \draw[->] (-1.2,0) -- (1.2,0);
    \draw[->] (0,-1.2) -- (0,1.2);
    \draw[thick] (0,0) circle (1);
    \coordinate[label=\iangle:$P$] (P) at (\iangle:1);
    \coordinate[label=below:$P_0$] (P0) at (P |- 0,0);
    \draw (0,0) -- (P);
    \draw (P) -- (P0);
    \fill[fill=gray,fill opacity=0.2]
      (0,0) -- (0:1) arc (0:\iangle:1) -- cycle;
    \filldraw[fill=gray,fill opacity=0.5]
      (0,0) -- (0:0.3) arc (0:\iangle:0.3) -- cycle;
    \node[right] at (\iangle/2:0.3) {\ang{\iangle}};
  \end{scope}
  % 右边的函数图
  \draw[->] (0,0) -- ({rad(210)}, 0);
  \draw[->] (0,-1.2) -- (0,1.2);
  \draw[thick, domain=0:rad(210)] plot (\x, {sin(\x r)})
    node[right] {$\sin x$};
  \foreach \t in {0, 90, 180} {
    \draw ({rad(\t)}, -0.05) -- ({rad(\t)}, 0.05);
    \node[below, outer sep=2pt, fill=white, font=\small]
      at ({rad(\t)}, 0) {\ang{\t}};
  }
  \foreach \y in {-1,1} {\draw (-0.05,\y) -- (0.05,\y);}
  \coordinate[label=above:$Q$] (Q) at ({rad(\iangle)}, {sin(\iangle)});
  \coordinate[label=below:$Q_0$] (Q0) at (Q |- 0,0);
  \draw (Q) -- (Q0);
  \draw[dashed] (P) -- (Q);
\end{tikzpicture}
\caption{正弦函数与单位圆(\textsf{TikZ} 实现)}
\label{fig:tikzsine}
\end{figure}
